\section{Motivation}
\hspace{0.5 cm}The precise control and manipulation of quantum systems has led to new developments in modern physics, particularly in scenarios involving extremely low temperatures. Achieving and measuring temperatures in the nanokelvin regime is now possible with ultra-cold atomic gases, but it presents both a technological and conceptual challenge, and new methods with atomic size probes are being developed. This emerging frontier has given rise to the field of \textit{\textbf{quantum thermometry}}, which seeks to determine the fundamental limits of temperature estimation in the quantum regime.\vspace{0.4 cm}

Although temperature is a familiar macroscopic concept, its formal definition becomes surprisingly difficult when applied to microscopic quantum systems. The existence of temperature, can be taken as the first steps in the axiomatic construction of thermodynamics, and all the well-established concepts like thermal equilibrium or efficiency of Carnot engines. However, when moving away from large Hamiltonian systems, these classical assumptions begin to fail. Quantum thermodynamics attempts to address this gap by redefining thermodynamic variables \cite{Binder2018}(such as heat, work, and temperature), or even reformulating the laws of thermodynamic, so that they remain meaningful at the quantum scale. In this context, quantum thermometry provides a theoretical framework for understanding how quantum systems can be used to infer temperature with optimal precision. One approach, that will be the one used in this work, involves weakly coupling a small quantum probe to a thermal system, allowing it to thermalise and then extracting temperature information from the probe alone. This is conceptually similar to how classical thermometers operate, but requires careful modeling of the quantum interactions involved. The framework also considers more general scenarios, including strong probe-system correlations and non-equilibrium dynamics, which can enhance sensitivity at ultra-low temperatures, but as a trade-off, one is required to develop a thorough comprehension of the dissipative interactions among them.\vspace{0.4 cm}

The precision of quantum thermometric protocols is typically bounded using tools from quantum estimation theory. These methods not only establish theoretical limits but also identify which observables or setups are most sensitive to temperature changes. Even when the optimal measurements are not experimentally accessible, sub-optimal but practical alternatives can be proposed.\vspace{0.4 cm}

Recent theoretical advances have shaped a unified view of quantum thermometry, enabling the comparison of diverse methods and the design of high-performance temperature probes. While experimental progress in nanoscale thermometry is rapidly advancing, this project will focus on some theoretical foundations and developments that brace the current understanding of temperature sensing in the quantum regime.

\section{Structure}
The thesis is organized as follows. In Chapter 2, we present the theoretical framework of quantum thermometry, introducing the concepts of Fisher Information, Quantum Fisher Information, and the general structure of equilibrium thermometers. Chapter 3 focuses on identifying the optimal quantum thermometer by maximizing the Quantum Fisher Information, and comparing this bound with the performance of realistic two-body spin network models. In Chapter 4, we extend the study to the dynamical regime, analyzing how thermometric quantities evolve during thermalization using open quantum system dynamics. Finally, in Chapter 5, we summarize the main conclusions and discuss possible applications and future directions.
